\documentclass[12pt]{exam}

\usepackage{amssymb}
\usepackage{mathtools}
\usepackage{algorithm}
\usepackage{float}  % Figure placement
\usepackage{minted}  % Code highlighting
\usepackage{tikz}  % Flow chart
\usepackage{lipsum}
\usepackage{xspace}
\usepackage{hyperref}
\usepackage{MnSymbol}
\usepackage{pgffor}
\usepackage{colortbl}
\usepackage{multirow}
\usepackage{array}


\hypersetup{
    colorlinks = true,
    linkcolor = blue,
    urlcolor  = blue,
    citecolor = blue,
    anchorcolor = blue
}

\newcommand{\hwheaderfooter}[3]{
\pagestyle{headandfoot}
\firstpageheadrule
\firstpageheader{#1}{#2}{#3}
\runningheader{#1}{#2}{#3}
\runningheadrule
\firstpagefooter{}{\thepage}{}
\runningfooter{}{\thepage}{}
}

\newcommand{\latex}{\LaTeX\xspace}

\newcommand{\stars}[1]{%
    \foreach \n in {1,...,#1}{%
        $\filledstar$%
    }%
}

\hwheaderfooter{Ching}{HW 8}{CSCI 406}


\begin{document}
\begin{center}
    \fbox{\fbox{\parbox{\textwidth - 0.2 in}{\centering

                {Instructions: Please note that handwritten assignments \textbf{will not be graded}. Use the
                    provided \latex template to complete your homework. Please do not alter the order or spacing of
                    questions (keep each question on its own page). When you submit to Gradescope, you must mark
                    which page(s) correspond to each question. \textbf{You may not receive credit for unmarked
                        questions}. \\ When including graphical figures, we encourage the use of tools such as \href{https://dreampuf.github.io/GraphvizOnline/}{graphviz} or packages like \href{https://www.overleaf.com/learn/latex/TikZ_package}{tikz} for simple and complex figures. However, these may be handwritten only if they are neat and legible (as defined by the grader). }
            }}}
\end{center}

\textbf{List any collaborators (besides TAs or professors) here:}

\pagebreak
\begin{questions}

    \question[55] [W9, \stars{5}] The traditional world chess championship is a match of 24 games. The current champion retains the title in case the match is a tie\footnote{This is no longer the format used to determine the champion. Instead, sets of progressively shorter games are played until one player has broken the tie.}. Each game ends in a win, loss, or draw (tie) where wins count as 1, losses as 0, and draws as $\frac{1}{2}$. The players take turns playing white and black with the champion playing white in the first game.

    Assume that $W(c)$, $D(c)$, and $L(c)$ return the probabilities that the champion wins, draws, or loses {\em one game} while playing color $c \in \{w, b\}$, respectively\footnote{Chess outcomes tend to be better playing white because white gets the first move.}. For example, $D(b)$ is the probability that the champion draws while playing black. You won't specifically need to use this, but convince yourself that $W(c) + D(c) + L(c) = 1.0$.

    Your task is to \textbf{calculate the champion's probability of retaining the title}.

    For the following questions, use the variable $g$ to represent the number of games remaining in the match, and the variable $i$ to represent the (possibly fractional) number of points that the champion needs to win to retain the title.

    \begin{parts}
        \part[5] What are the inputs to the \textbf{subproblem instance}? (\textbf{Hint:} your answer will be in terms of the variables described above.)

        The inputs to the subproblem instance are the number of games remaining in the match $g$, the number of points that the champion needs to win to retain the title $i$, and the color the champion is playing in the current game $c$.

        For example, the subproblem instance $P(g=24, i=12, c=w)$ represents the original problem instance where the champion needs to win 12 games to retain the title, where $P(g, i, c)$ is the probability of the champion retaining the title.

        \part[5] Which subproblem instances are the \textbf{base cases}? What are their values? (\textbf{Hint:} think about subproblem instances where the probability of the champion retaining the title are not dependent on any of the win/draw/loss probabilities.)

        The base cases are when the champion has already won or lost the title. The base cases are:
        \begin{itemize}
            \item If the champion has already won the title (needs to win 0 games of the remaining $(i=0)$), the probability of retaining the title is 1.0.
            \item If the champion has already lost the title (remaining games is less than games needed to win $(g<i)$), the probability of retaining the title is 0.0.
        \end{itemize}

        \pagebreak
        \part[15] Given your answers to the above questions, develop notation and write a \textit{formula} the \textbf{recurrence relation}. Assume that the function $C(g)$ returns the color the champion is playing in game $g$.

        The recurrence relation is:
        \begin{align*}
            P(g, i, c) = \begin{cases}
                             1.0                                            & \text{if } i <= 0 \\
                             0.0                                            & \text{if } g < i  \\
                             W(c) \cdot P(g-1, i-1, \text{next}(c))                             \\
                             \indent + D(c) \cdot P(g-1, i-\frac{1}{2}, \text{next}(c))         \\
                             \indent + L(c) \cdot P(g-1, i, \text{next}(c)) & \text{otherwise}
                         \end{cases}
        \end{align*}
        where $\text{next}(c)$ is the color the champion will play in the next game, $\text{next}(w) = b$ and $\text{next}(b) = w$.

        \part[5] How many \textbf{dimensions} will the dynamic programming table (used in bottom-up DP) have? What will each dimension of your table represent? What is the domain of each dimension?

        The dynamic programming table will have three dimensions, one for each varible $g$, $i$, and $c$. The domain of each dimension is as follows:
        \begin{itemize}
            \item $g \in [0, 24]$.
            \item $i \in [0, 12]$.
            \item $c \in \{w, b\}$.
        \end{itemize}

        \part[5] In \textbf{which cell of your bottom-up DP table will you find the answer} to the posed question (the champion's probability of retaining the title)?

        The answer to the posed question will be found in the cell $P(24, 12, w)$.

        \part[10] What \textbf{iteration order} is required to compute the bottom-up DP table?

        The iteration order required to compute the bottom-up DP table is to iterate over the number of games remaining $g$ from 0 to 24, the number of games the champion needs to win $i$ from 0 to 12, and the color the champion is playing $c$ from \{w, b\}.

        \part[10] Imagine that the championship match was a variable number of games $n$. What is the \textbf{complexity of this bottom-up DP algorithm in terms of $n$}?

        The complexity of this bottom-up DP algorithm in terms of $n$ is $O(n^2)$. This is because the algorithm iterates over the number of games remaining $g$ from 0 to $n$ and the number of games the champion needs to win $i$ from 0 to $\frac{n}{2}$.

    \end{parts}
    \clearpage

    \question[45] [W10, \stars{4}] A palindrome is a string that is equal to itself when reversed. Formally, a string $s_1s_2{\cdots}s_n$ is a palindrome if $s_1s_2{\cdots}s_{n-1}s_n = s_ns_{n-1}{\cdots}s_2s_1$. According to wikipedia, the 19-letter Finnish word saippuakivikauppias is the
    longest palindromic word in everyday use.

    Given a string, $S$, you want to find the \textbf{longest palindromic subsequence}. That is, the longest subsequence of $S$ where the subsequence is a palindrome. Remember, a subsequence is \textit{not} necessarily continuous.

    \begin{parts}
        \part[5] Provide two examples of longest palindromic subsequences for the following string: \texttt{COMPUTERSCIENCE}.

        \begin{itemize}
            \item \texttt{CETECE}
            \item \texttt{COMSOMC}
        \end{itemize}

        \part[5] What are the inputs to the \textbf{subproblem instance}?

        The inputs to the subproblem instance are the start and end indices of the string $S$, represented as $i$ and $j, 0 \leq i \leq j \leq n$, where $n$ is the length of the string $S$.

        \part[5] Which subproblem instances are the \textbf{base cases}? What are their values?

        The base cases are:
        \begin{itemize}
            \item If $i = j$, the length of the longest palindromic subsequence is 1.
            \item If $i + 1 = j$ and $S[i] = S[j]$, the length of the longest palindromic subsequence is 2.
        \end{itemize}

        \part[15] Given your answers to the above questions, develop notation and write a \textit{formula} the \textbf{recurrence relation}.

        The recurrence relation is:
        \begin{align*}
            LPS(i, j) = \begin{cases}
                            1                              & \text{if } i = j                              \\
                            2                              & \text{if } i + 1 = j \text{ and } S[i] = S[j] \\
                            \max(LPS(i+1, j), LPS(i, j-1)) & \text{otherwise}
                        \end{cases}
        \end{align*}

        \pagebreak
        \part[15]Write a \textbf{``top-down'' DP solution} that augments the recursive algorithm above with memoization. Store intermediate results in a dictionary/array and check to see if a value has already been determined before computing it again. If the value has already been computed, then return it instead of making the recursive call(s). \textbf{Present your code using the minted \latex library. You shouldn't need more than 10 lines.}

        \begin{minted}{python}
def longest_palindromic_substring(S):
    memo = {}
    def dp(i, j):
        if i > j:
            return 0
        if i == j:
            return 1
        if (i, j) not in memo:
            if S[i] == S[j]:
                memo[i, j] = dp(i + 1, j - 1) + 2
            else:
                memo[i, j] = max(dp(i, j - 1), dp(i + 1, j))
        return memo[i, j]
    return dp(0, len(S) - 1)
        \end{minted}

    \end{parts}

    % close the document
\end{questions}
\end{document}
