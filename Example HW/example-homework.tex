\documentclass[12pt]{exam}

\usepackage{amssymb}
\usepackage{mathtools}
\usepackage{algorithm}
\usepackage{float}  % Figure placement
\usepackage{minted}  % Code highlighting
\usepackage{tikz}  % Flow chart
\usepackage{lipsum}
\usepackage{xspace}
\usepackage{hyperref}
\usepackage{MnSymbol}

\hypersetup{
    colorlinks = true,
    linkcolor = blue,
    urlcolor  = blue,
    citecolor = blue,
    anchorcolor = blue
}

\newcommand{\hwheaderfooter}[3]{
\pagestyle{headandfoot}
\firstpageheadrule
\firstpageheader{#1}{#2}{#3}
\runningheader{#1}{#2}{#3}
\runningheadrule
\firstpagefooter{}{\thepage}{}
\runningfooter{}{\thepage}{}
}

\newcommand{\latex}{\LaTeX\xspace}


\hwheaderfooter{Example Homework}{CHING}{CSCI 406}


\begin{document}
\begin{center}
  \fbox{\fbox{\parbox{\textwidth - 0.2 in}{\centering

        {Instructions: Please note that handwritten assignments \textbf{will not be graded}. Use the
          provided \latex template to complete your homework. Please do not alter the order or spacing of
          questions (keep each question on its own page). When you submit to Gradescope, you must mark
          which page(s) correspond to each question. \textbf{You may not receive credit for unmarked
            questions}. \\ When including graphical figures, we encourage the use of tools such as \href{https://dreampuf.github.io/GraphvizOnline/}{graphviz} or packages like \href{https://www.overleaf.com/learn/latex/TikZ_package}{tikz} for simple and complex figures. However, these may be handwritten only if they are neat and legible (as defined by the grader). }\\

      }}}
\end{center}

\begin{center}
  \fbox{\fbox{\parbox{\textwidth - 0.2 in}{
        \begin{center}
          \textbf{Important Homework Information}
        \end{center}

        \begin{itemize}
          \setlength\itemsep{0em}
          \item Each homework assignment is worth 100 points.
          \item All homeworks are weighted evenly.
          \item Each homework will have some easier questions from the current week's content,
                and some harder questions from the previous week's content.
          \item It is recommended that you work on the questions from the previous week's content
                throughout the week. The questions related to the current week's content should be
                completed throughout the week.
          \item The point value of each question is listed next to the question.
          \item In brackets after the point amount are two values: the week number the question
                is related to, and a difficulty score in 1-5 $\filledstar$s. These are meant to help
                you determine which problems will likely take more time. The scores are relative to
                the questions in that particular homework.
          \item Include one question per page. (It's ok to keep parts on the same page.)
        \end{itemize}

      }}}
\end{center}

\begin{questions}

  \question For the following questions, select whether the statement is true or false,
  and write a \textit{brief} explanation of your reasoning.

  \begin{parts}
    \part[10] [W0, $\filledstar$] There will be a homework assignment most weeks.

    % Replace \square with \blacksquare for the option you would like to select.
    $\blacksquare$ True $\square$ False

    \part[10] [W0, $\filledstar$] All homework assignments are due on Saturday night.

    % Replace \square with \blacksquare for the option you would like to select.
    $\blacksquare$ True $\square$ False

    \part[10] [W3, $\filledstar\filledstar\filledstar\filledstar\filledstar$]
    This problem is marked as a problem about content from Week 3 with minimal difficulty.

    % Replace \square with \blacksquare for the option you would like to select.
    $\square$ True $\blacksquare$ False

  \end{parts}

  \clearpage

  \question[70] [W0, $\filledstar\filledstar\filledstar$]
  Design a dictionary data structure in which search, insertion,
  and deletion can all be processed in $\mathcal{O}(1)$ time in the worst
  case. You may assume the set elements are integers drawn from a finite set
  $1, 2, \dots , n$, and initialization can take $\mathcal{O}(n)$ time.

  % Numbered list
  \begin{enumerate}
    \item Allocate an array of length $n$.
    \item Search for key $i$ can be performed by looking if an element exist at index $i$.
    \item Insert and remopve at key $i$ is place/remove element at index $i$.
    \item Insert, remove, and index are all $\mathcal{O}(1)$. so all operations are constant time.
  \end{enumerate}

  % close the document
\end{questions}
\end{document}
